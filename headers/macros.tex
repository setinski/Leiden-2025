% Jan 2013, edited by DD
% Oct 2004, edited by OR
%%Version Feb 3 2001; last edited by SA
% Feb 13, 2001: LT adds following macros:
%  \logred, \conl, \genclass, \cogenclass, \size,
%  \floor, \ceil, \comp, \ffigure, \ffigureh
%  \iff, \implies, \sigmatwo, \sigmathree, \pitwo,
%  \scand, \scor, \scnot, \scyes
%Feb20 SA added \sig and \pip for boldface polynomial hierarchy
%% Macros for complexity theory notation
%% Complexity classes



%% Notation for integers, natural numbers, reals, fractions, sets, cardinalities
%%and so on

\newcommand\B{\{0,1\}}      % boolean alphabet  use in math mode
\newcommand\Bs{\{0,1\}^*}   % B star            use in math mode
\newcommand\true{\mbox{\sc True}}
\newcommand\false{\mbox{\sc False}}
\DeclareRobustCommand{\fracp}[2]{{#1 \overwithdelims()#2}}
\DeclareRobustCommand{\fracb}[2]{{#1 \overwithdelims[]#2}}
\newcommand{\marginlabel}[1]%
{\mbox{}\marginpar{\it{\raggedleft\hspace{0pt}#1}}}
\newcommand\card[1]{\left| #1 \right|} %cardinality of set S; usage \card{S}
\newcommand{\comp}[1]{\overline{#1}}
\newcommand{\tran}[1]{{#1}^T}


%% Various things to write in small caps
\def\scand{\mbox{\sc and}}
\def\scor{\mbox{\sc or}}
\def\scnot{\mbox{\sc not}}
\def\scyes{\mbox{\sc yes}}
\def\scno{\mbox{\sc no}}
\def\scmaybe{\mbox{\sc maybe}}

% new epsilons
\newcommand{\eps}{\varepsilon}
\renewcommand{\epsilon}{\varepsilon}

\def\lecturegg{7}

% GENERAL COMPUTING STUFF
\newcommand{\bit}{\ensuremath{\set{0,1}}}
\newcommand{\pmone}{\ensuremath{\set{-1,1}}}

% asymptotic stuff
\DeclareMathOperator{\poly}{poly}
\DeclareMathOperator{\polylog}{polylog}
\DeclareMathOperator{\negl}{negl}
\newcommand{\Otil}{\ensuremath{\tilde{O}}}

% probability/distribution stuff
\DeclareMathOperator*{\E}{\mathbb{E}}
\DeclareMathOperator*{\Var}{Var}

% assorted
\DeclareMathOperator*{\wt}{wt}

% hash functions
\newcommand{\calH}{\ensuremath{\mathcal{H}}}
\newcommand{\calX}{\ensuremath{\mathcal{X}}}
\newcommand{\calY}{\ensuremath{\mathcal{Y}}}
\newcommand{\calL}{\ensuremath{\mathcal{L}}}
\newcommand{\calB}{\ensuremath{\mathcal{B}}}

\newcommand{\compind}{\ensuremath{\stackrel{c}{\approx}}}
\newcommand{\statind}{\ensuremath{\stackrel{s}{\approx}}}
\newcommand{\perfind}{\ensuremath{\equiv}}

% font for general-purpose algorithms
\newcommand{\algo}[1]{\ensuremath{\mathsf{#1}}}

% font for complexity classes
\newcommand{\class}[1]{\ensuremath{\mathsf{#1}}}

% complexity classes and languages
\renewcommand{\P}{\class{P}}
\newcommand{\BPP}{\class{BPP}}
\newcommand{\NP}{\class{NP}}
\newcommand\RP{\class{RP}}
\newcommand\ZPP{\class{ZPP}}
\newcommand{\coNP}{\class{coNP}}
\newcommand{\AM}{\class{AM}}
\newcommand{\coAM}{\class{coAM}}
\newcommand{\IP}{\class{IP}}
\newcommand{\SZK}{\class{SZK}}
\newcommand{\NISZK}{\class{NISZK}}
\newcommand{\NICZK}{\class{NICZK}}
\newcommand{\PPP}{\class{PPP}}
\newcommand{\PPAD}{\class{PPAD}}

% \newcommand\SIGMA2{\class{$\Sigma_2$}}
% \newcommand\SIGMA3{\class{$\Sigma_3$}}
% \newcommand\PI2{\mbox{\bf $\Pi_2$}\xspace}
\newcommand\PH{\mbox{\bf PH}\xspace}
\newcommand\PSPACE{\mbox{\bf PSPACE}\xspace}
\newcommand\NPSPACE{\mbox{\bf NPSPACE}\xspace}
\newcommand\DL{\mbox{\bf L}\xspace}
\newcommand\NL{\mbox{\bf NL}\xspace}
\newcommand\coNL{\mbox{\bf coNL}\xspace}
\newcommand\sharpP{\mbox{\#{\bf P}}\xspace}
\newcommand\parityP{\mbox{$\oplus$ {\bf P}}\xspace}
\newcommand\PCP{\mbox{\bf PCP}}
\newcommand\DTIME{\mbox{\bf DTIME}}
\newcommand\NTIME{\mbox{\bf NTIME}}
\newcommand\DSPACE{\mbox{\bf SPACE}\xspace}
\newcommand\NSPACE{\mbox{\bf NSPACE}\xspace}
\newcommand\CNSPACE{\mbox{\bf coNSPACE}\xspace}
\newcommand\EXPTIME{\mbox{\bf EXPTIME}\xspace}
\newcommand\NEXPTIME{\mbox{\bf NEXPTIME}\xspace}
\newcommand\GENCLASS{\mbox{$\cal C$}\xspace}
\newcommand\coGENCLASS{\mbox{\bf co$\cal C$}\xspace}
\newcommand\SIZE{\mbox{\bf SIZE}\xspace}
\newcommand\sig{\mathbf \Sigma}
\newcommand\pip{\mathbf \Pi}

%%Computational problems
\newcommand\sat{\mbox{SAT}\xspace}
\newcommand\tsat{\mbox{3SAT}\xspace}
\newcommand\tqbf{\mbox{TQBF}\xspace}


\newcommand{\yes}{\ensuremath{\text{YES}}}
\newcommand{\no}{\ensuremath{\text{NO}}}

\newcommand{\Piyes}{\ensuremath{\Pi^{\yes}}}
\newcommand{\Pino}{\ensuremath{\Pi^{\no}}}

% macros for matrices and vectors
\newcommand{\matA}{\ensuremath{\mathbf{A}}}
\newcommand{\matB}{\ensuremath{\mathbf{B}}}
\newcommand{\matC}{\ensuremath{\mathbf{C}}}
\newcommand{\matD}{\ensuremath{\mathbf{D}}}
\newcommand{\matE}{\ensuremath{\mathbf{E}}}
\newcommand{\matF}{\ensuremath{\mathbf{F}}}
\newcommand{\matG}{\ensuremath{\mathbf{G}}}
\newcommand{\matH}{\ensuremath{\mathbf{H}}}
\newcommand{\matI}{\ensuremath{\mathbf{I}}}
\newcommand{\matJ}{\ensuremath{\mathbf{J}}}
\newcommand{\matK}{\ensuremath{\mathbf{K}}}
\newcommand{\matL}{\ensuremath{\mathbf{L}}}
\newcommand{\matM}{\ensuremath{\mathbf{M}}}
\newcommand{\matN}{\ensuremath{\mathbf{N}}}
\newcommand{\matO}{\ensuremath{\mathbf{O}}}
\newcommand{\matP}{\ensuremath{\mathbf{P}}}
\newcommand{\matQ}{\ensuremath{\mathbf{Q}}}
\newcommand{\matR}{\ensuremath{\mathbf{R}}}
\newcommand{\matS}{\ensuremath{\mathbf{S}}}
\newcommand{\matT}{\ensuremath{\mathbf{T}}}
\newcommand{\matU}{\ensuremath{\mathbf{U}}}
\newcommand{\matV}{\ensuremath{\mathbf{V}}}
\newcommand{\matW}{\ensuremath{\mathbf{W}}}
\newcommand{\matX}{\ensuremath{\mathbf{X}}}
\newcommand{\matY}{\ensuremath{\mathbf{Y}}}
\newcommand{\matZ}{\ensuremath{\mathbf{Z}}}
\newcommand{\matzero}{\ensuremath{\mathbf{0}}}

\newcommand{\veca}{\ensuremath{\mathbf{a}}}
\newcommand{\vecb}{\ensuremath{\mathbf{b}}}
\newcommand{\vecc}{\ensuremath{\mathbf{c}}}
\newcommand{\vecd}{\ensuremath{\mathbf{d}}}
\newcommand{\vece}{\ensuremath{\mathbf{e}}}
\newcommand{\vecf}{\ensuremath{\mathbf{f}}}
\newcommand{\vecg}{\ensuremath{\mathbf{g}}}
\newcommand{\vech}{\ensuremath{\mathbf{h}}}
\newcommand{\veck}{\ensuremath{\mathbf{k}}}
\newcommand{\vecm}{\ensuremath{\mathbf{m}}}
\newcommand{\vecp}{\ensuremath{\mathbf{p}}}
\newcommand{\vecq}{\ensuremath{\mathbf{q}}}
\newcommand{\vecr}{\ensuremath{\mathbf{r}}}
\newcommand{\vecs}{\ensuremath{\mathbf{s}}}
\newcommand{\vect}{\ensuremath{\mathbf{t}}}
\newcommand{\vecu}{\ensuremath{\mathbf{u}}}
\newcommand{\vecv}{\ensuremath{\mathbf{v}}}
\newcommand{\vecw}{\ensuremath{\mathbf{w}}}
\newcommand{\vecx}{\ensuremath{\mathbf{x}}}
\newcommand{\vecy}{\ensuremath{\mathbf{y}}}
\newcommand{\vecz}{\ensuremath{\mathbf{z}}}
\newcommand{\veczero}{\ensuremath{\mathbf{0}}}

\renewcommand{\vec}[1]{\ensuremath{\mathbf{#1}}}

\DeclareMathOperator{\diag}{diag}

% blackboard symbols

\newcommand{\C}{\ensuremath{\mathbb{C}}}
\newcommand{\D}{\ensuremath{\mathbb{D}}}
\newcommand{\F}{\ensuremath{\mathbb{F}}}
\newcommand{\G}{\ensuremath{\mathbb{G}}}
\newcommand{\J}{\ensuremath{\mathbb{J}}}
\newcommand{\N}{\ensuremath{\mathbb{N}}}
\newcommand{\Q}{\ensuremath{\mathbb{Q}}}
\newcommand{\R}{\ensuremath{\mathbb{R}}}
\newcommand{\Z}{\ensuremath{\mathbb{Z}}}

\newcommand{\Zt}{\ensuremath{\Z_t}}
\newcommand{\Zp}{\ensuremath{\Z_p}}
\newcommand{\Zq}{\ensuremath{\Z_q}}
\newcommand{\ZN}{\ensuremath{\Z_N}}
\newcommand{\Zps}{\ensuremath{\Z_p^*}}
\newcommand{\ZNs}{\ensuremath{\Z_N^*}}
\newcommand{\JN}{\ensuremath{\J_N}}
\newcommand{\QRN}{\ensuremath{\mathbb{QR}_N}}

% Euclidean Ball
\newcommand{\ball}{\mathcal{B}_2}

% "left-right" pairs of symbols

% inner product
\newcommand{\pr}[2]{\langle{#1, #2}\rangle}
\newcommand{\prfit}[2]{\left\langle{#1, #2}\right\rangle}

% absolute value
\newcommand{\absfit}[1]{\left\lvert{#1}\right\rvert}
% a set
\newcommand{\set}[1]{\{{#1}\}}
\newcommand{\setfit}[1]{\left\{{#1}\right\}}
% parens
\newcommand{\parens}[1]{({#1})}
\newcommand{\parensfit}[1]{\left({#1}\right)}
% tuple, alias for parens
\newcommand{\tuple}[1]{\parens{#1}}
\newcommand{\tuplefit}[1]{\parensfit{#1}}
% square brackets
\newcommand{\bracks}[1]{[{#1}]}
\newcommand{\bracksfit}[1]{\left[{#1}\right]}
% rounding off
\newcommand{\round}[1]{\lfloor{#1}\rceil}
\newcommand{\roundfit}[1]{\left\lfloor{#1}\right\rceil}
% floor function
\newcommand{\floor}[1]{\lfloor{#1}\rfloor}
\newcommand{\floorfit}[1]{\left\lfloor{#1}\right\rfloor}
% ceiling function
\newcommand{\ceil}[1]{\lceil{#1}\rceil}
\newcommand{\ceilfit}[1]{\left\lceil{#1}\right\rceil}
% length of some vector, element
\newcommand{\length}[1]{\lVert{#1}\rVert}
\newcommand{\lengthfit}[1]{\left\lVert{#1}\right\rVert}
\newcommand{\eqdef}{\mathbin{\stackrel{\rm def}{=}}}

%transpose
\newcommand{\T}{\mathsf{T}}

% gram schmidt and dual bases
\newif\ifleo\leotrue
%\newif\ifleo\leofalse

\ifleo

\newcommand{\gs}[1]{{\vec #1}^\star}
\newcommand{\du}[1]{{\vec #1}^{\vee}}
\newcommand{\dl}[1]{{#1}^{\vee}}
\newcommand{\dgs}[1]{{\vec #1}^{\star}}
\newcommand{\mat}[1]{\vec{#1}}
\newcommand{\gsm}[1]{{\mat{#1}}^\star}
\newcommand{\dm}[1]{\mat{#1}^{\vee}}
\newcommand{\rdm}[1]{\mat{#1}^{\wedge}}%Reverse Dual
\newcommand{\dmt}[1]{\mat{#1}^{\vee \T}}
\newcommand{\dgsm}[1]{\mat{#1}^{*\vee}}

\else

\newcommand{\gs}[1]{\widetilde{\mathbf #1}}
\newcommand{\rgs}[1]{\accentset{\curvearrowleft}{\mathbf #1}}
\newcommand{\du}[1]{{\mathbf #1}^{*}}
\newcommand{\dl}[1]{{#1}^{*}}
\newcommand{\dgs}[1]{\widetilde{\mathbf #1}^{*}}
\newcommand{\rdgs}[1]{\accentset{\curvearrowleft}{\mathbf #1}^{*}}
\newcommand{\mat}[1]{\vec{#1}}
\newcommand{\gsm}[1]{\widetilde{\mat{#1}}}
\newcommand{\dm}[1]{\mat{#1}^*}
\newcommand{\rdm}[1]{{}^*\mat{#1}}%Reverse Dual
\newcommand{\dmt}[1]{\mat{#1}^{*\T}}
\newcommand{\dgsm}[1]{\widetilde{\mat{#1}}^{*}}

\fi

\newcommand{\vc}[1]{\vec{#1}}
\newcommand{\mt}[1]{\mat{#1}}

%\DeclareMathOperator*{\supp}{supp}
\DeclareMathOperator*{\conv}{conv}
\def\sym{\bigtriangleup}
\DeclareMathOperator*{\TVD}{d_{TV}}
\DeclareMathOperator*{\width}{width}
\DeclareMathOperator*{\argmax}{arg\,max}
\DeclareMathOperator*{\argmin}{arg\,min}
\DeclareMathOperator*{\cov}{cov}
\DeclareMathOperator{\SVP}{SVP}
\DeclareMathOperator{\HermiteSVP}{HermiteSVP}
\DeclareMathOperator{\CVP}{CVP}

%%% new commands -- norms, operators, blah blah.
\DeclarePairedDelimiter\norm{\lVert}{\rVert}
\DeclarePairedDelimiter\abs{\lvert}{\rvert}%
\newcommand{\enc}[1]{\langle #1 \rangle}
\makeatletter
\def\imod#1{\allowbreak\mkern10mu({\operator@font mod}\,\,#1)}
\makeatother

%%%% some other macros that can be cleaned up
\newcommand{\parl}[1]{\mathcal{P}(#1)}
\def\diag{\mathrm{diag}}
\def\Id{\mathrm{Id}}
\def\sph{\mathbb{S}}
\def\suchthat{\;:\;}
\def\psd{\succcurlyeq}
\def\minuszero{\setminus \set{\vec 0}}
\def\diam{\mathrm{Diam}}
\def\core{\mathcal{C}}
\def\st{~\mathrm{s.t.}~}
\def\linsp{\mathrm{span}}
\def\Gram{\mathrm{Gram}}
\def\vol{\mathrm{vol}}
\def\det{\mathrm{det}}
\def\supp{\mathrm{support}}
\def\lat{\mathcal{L}}
\def\aff{{\rm aff}}
\def\ext{{\rm ext}}
\def\cone{{\rm cone}}
\def\bd{{\rm bd}}
\def\relbd{{\rm relbd}}
\def\ker{{\rm ker}}
\def\d{{\rm d}}

\DeclareMathOperator*{\unif}{uniform}

%\newcommand\to{\rightarrow}
\newcommand{\from}{:}
\newcommand\xor{\oplus}
\newcommand\bigxor{\bigoplus}
\newcommand{\logred}{\leq_{\log}}
\def\iff{\Leftrightarrow}
\def\implies{\Rightarrow}

%% macros to write pseudo-code
\floatstyle{ruled}
\newfloat{procedure}{h!}{loa}
\floatname{procedure}{Procedure}

\newif\ifnewprog\newprogfalse

\ifnewprog

\newlength{\pgmtab}  %  \pgmtab is the width of each tab in the
\setlength{\pgmtab}{1em}  %  program environment
 \newenvironment{program}{\renewcommand{\baselinestretch}{1}%
\begin{tabbing}\hspace{0em}\=\hspace{0em}\=%
\hspace{\pgmtab}\=\hspace{\pgmtab}\=\hspace{\pgmtab}\=\hspace{\pgmtab}\=%
\hspace{\pgmtab}\=\hspace{\pgmtab}\=\hspace{\pgmtab}\=\hspace{\pgmtab}\=%
\+\+\kill}{\end{tabbing}\renewcommand{\baselinestretch}{\intl}}
\newcommand {\BEGIN}{{\bf begin\ }}
\newcommand {\ELSE}{{\bf else\ }}
\newcommand {\IF}{{\bf if\ }}
\newcommand {\FOR}{{\bf for\ }}
\newcommand {\TO}{{\bf to\ }}
\newcommand {\DO}{{\bf do\ }}
\newcommand {\WHILE}{{\bf while\ }}
\newcommand {\ACCEPT}{{\bf accept}}
\newcommand {\REJECT}{\mbox{\bf reject}}
\newcommand {\THEN}{\mbox{\bf then\ }}
\newcommand {\END}{{\bf end}}
\newcommand {\RETURN}{\mbox{\bf return\ }}
\newcommand {\HALT}{\mbox{\bf halt}}
\newcommand {\REPEAT}{\mbox{\bf repeat\ }}
\newcommand {\UNTIL}{\mbox{\bf until\ }}
\newcommand {\TRUE}{\mbox{\bf true\ }}
\newcommand {\FALSE}{\mbox{\bf false\ }}
\newcommand {\FORALL}{\mbox{\bf for all\ }}
\newcommand {\DOWNTO}{\mbox{\bf down to\ }}

\fi

%homework stuff
\definecolor{hintgray}{gray}{0.40}

%formatting commands
\newcounter{lecnum}
\newcommand{\lecture}[3]{%
%\pagestyle{myheadings}%
%\thispagestyle{plain}%
\newpage %
\setcounter{lecnum}{#1}%
\setcounter{page}{1}%
%\rule{\linewidth}{1mm}

 \thispagestyle{fancy}
 \headheight 45pt
 \lhead{\bf{Leiden, Spring 2024 \\ Lattices and their Algorithms} \vspace{.5em}\\ \hspace{1em}}
 \chead{{\Large \bf Lecture #1} \vspace{.5em} \\ \hrule height .5mm \hfill \\ {\bf \Large #2}}
 \rhead{\bf{Lecturers: L. Ducas \\ T.A.: #3} \vspace{.5em} \\ \hspace{1em}}
 \renewcommand{\headrulewidth}{1.2pt}
 \hspace{1em} \vspace{6pt}
}

\newcommand{\exercisedesc}[2]{%
 \newpage %
 \setcounter{lecnum}{#1}%
 \setcounter{page}{1}%

 \pagestyle{fancy}
 \headheight 30pt
 \lhead{\bf{Spring 2024 \\ Lattices and their Algorithms}}
 \chead{\bf{\large{Exercise #1 \\ Discussion on #2}}}
 \rhead{\bf{L. Ducas \\ Leiden}}
 \renewcommand{\headrulewidth}{1.2pt}
 \setlength{\headsep}{10pt}
}

\newcommand{\homeworkdesc}[2]{%
 \newpage %
 \setcounter{lecnum}{#1}%
 \setcounter{page}{1}%

 \pagestyle{fancy}
 \headheight 30pt
 \lhead{\bf{Spring 2024 \\Lattices and their Algorithms}}
 \chead{\bf{\large{Homework #1 \\ Due #2}}}
 \rhead{\bf{L. Ducas \\ Leiden}}
 \renewcommand{\headrulewidth}{1.2pt}
 \setlength{\headsep}{10pt}
}

\newcommand{\homeworksol}[1]{%
 \newpage %
 \setcounter{lecnum}{#1}%
 \setcounter{page}{1}%

 \pagestyle{fancy}
 \headheight 30pt
 \lhead{\bf{Spring 2024 \\ Lattices and their Algorithms}}
 \chead{\bf{\large{Partial Solutions to \\ Homework #1}}}
 \rhead{\bf{Leiden}}
 \renewcommand{\headrulewidth}{1.2pt}
 \setlength{\headsep}{10pt}
}

\newcommand{\exercisesol}[1]{%
 \newpage %
 \setcounter{lecnum}{#1}%
 \setcounter{page}{1}%

 \pagestyle{fancy}
 \headheight 30pt
 \lhead{\bf{Spring 2024 \\Lattices and their Algorithms}}
 \chead{\bf{\large{Solutions to \\ Exercise #1}}}
 \rhead{\bf{Leiden}}
 \renewcommand{\headrulewidth}{1.2pt}
 \setlength{\headsep}{10pt}
}

\newcommand{\exam}{%
 \newpage %
 \setcounter{page}{1}%

 \pagestyle{fancy}
 \headheight 30pt
 \lhead{\bf{Spring 2024 \\ Lattices and their Algorithms}}
 \chead{\bf{\large{Final Exam}}}
 \rhead{\bf{L. Ducas \\ Leiden}}
 \renewcommand{\headrulewidth}{1.2pt}
 \setlength{\headsep}{10pt}
}

% wide pages
\setlength{\topmargin}{-0.5in}
\setlength{\textwidth}{6.5in} % can be up to 6.5
\setlength{\textheight}{9in}
\setlength{\evensidemargin}{-.1in}
\setlength{\oddsidemargin}{-.1in}


%%%%%%%%%%%%%%%%%%%%%%%%%%%%%%%%%%%%%%%%%%%%%%%%%%%%%%%%%%%%%%%%%%%%%%%%%%%
%%%%%%%%%%%%%%%%%%%%%%%%%%%%%%%%%%%%%%%%%%%%%%%%%%%%%%%%%%%%%%%%%%%%%%%%%%%

\newlength{\tpush}
\setlength{\tpush}{2\headheight}
\addtolength{\tpush}{\headsep}

%       Usage: \htitle{title}{datelec}{dateout}
\newcommand{\htitle}[4]{\noindent\vspace*{-\tpush}\newline\parbox{\textwidth}
{U.C. Berkeley --- \coursenum : \coursename \hfill #1 \newline
\courseprof \hfill #2 \newline
\mbox{#4}\hfill Last revised #3 \newline
\mbox{}\hrulefill\mbox{}}\vspace*{1ex}\mbox{}\newline
\bigskip
\begin{center}{\Large\bf #1}\end{center}
\bigskip}

%       Usage: \handout{title}{datelec}{dateout}{scribe}
\newcommand{\handout}[4]{\thispagestyle{empty}
 \markboth{Notes for #1: #2}{Notes for #1: #2}
 \pagestyle{myheadings}\htitle{#1}{#2}{#3}{#4}}

\newcommand{\htitlewithouttitle}[2]{\noindent\vspace*{-\tpush}\newline\parbox{6.5in}{\coursenumber : \coursetitle
\hfill Columbia University\newline
Handout #1\hfill#2\vspace*{-.5ex}\newline
\mbox{}\hrulefill\mbox{}}\vspace*{1ex}\mbox{}\newline}

\newcommand{\handoutwithouttitle}[2]{\thispagestyle{empty}
 \markboth{Handout #1}{Handout #1}
 \pagestyle{myheadings}\htitlewithouttitle{#1}{#2}}


%%%%%%%%%%%%%%%%%%%%%%%%%%%%%%%%%%%%%%%%%%%%%%%%%%%%%%%%%%%%%%%%%
%%% Commands to include figures


%% PSfigure

\newcommand{\PSfigure}[3]{\begin{figure}[t]
  \centerline{\vbox to #2 {\vfil \includegraphics[height=#2]{#1} }}
  \caption{#3}\label{#1}
  \end{figure}}
\newcommand{\twoPSfigures}[5]{\begin{figure*}[t]
  \centerline{%
    \hfil
    \begin{tabular}{c}
        \vbox to #3 {\vfil\includegraphics[height=#3]{#1}} \\ (a)
    \end{tabular}
    \hfil\hfil\hfil
    \begin{tabular}{c}
        \vbox to #3 {\vfil\includegraphics[height=#3]{#1}} \\ (b)
    \end{tabular}
    \hfil}
  \caption{#4}
  \label{#5}
%  \sublabel{#1}{(a)}
%  \sublabel{#2}{(b)}
  \end{figure*}}


\newcounter{fignum}

% fig
%command to insert figure. usage \fig{name}{h}{caption}
%where name.eps is the postscript file and h is the height in inches
%The figure is can be referred to using \ref{name}
\newcommand{\fig}[3]{%
\begin{minipage}{\textwidth}
\centering\includegraphics[height=#2]{#1}
\caption{#3} \label{#1}
\end{minipage}
}%


% ffigure
% Usage: \ffigure{name of file}{height}{caption}{label}
\newcommand{\ffigure}[4]{\begin{figure}
  \centerline{\vbox to #2 {\vfil \includegraphics[height=#2]{#1} }}
  \caption{#3}\label{#4}
  \end{figure}}

% ffigureh
% Usage: \ffigureh{name of file}{height}{caption}{label}
\newcommand{\ffigureh}[4]{\begin{figure}[!h]
  \centerline{\vbox to #2 {\vfil \includegraphics[height=#2]{#1} }}
  \caption{#3}\label{#4}
  \end{figure}}

\newcommand{\solution}[1]{%
\ifsolutions
\vspace{-3ex}
\noindent\textbf{Solution:} {\itshape #1}
\vspace{3ex}
\fi
}

\newcommand{\hint}[1]{\textcolor{hintgray}{(Hint:~ #1)}}

